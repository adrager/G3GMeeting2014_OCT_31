\documentclass{beamer}

\usetheme[secheader]{Boadilla}
\setbeamertemplate{footline} {
  %\leavevmode%
  \hbox{%
  \begin{beamercolorbox}[wd=.5\paperwidth,ht=2.25ex,dp=1ex,left]{author in head/foot}%
    \usebeamerfont{author in head/foot}\hspace*{2ex}\insertshortauthor~~(adraeger@cern.ch)
  \end{beamercolorbox}%
  \begin{beamercolorbox}[wd=.5\paperwidth,ht=2.25ex,dp=1ex,right]{date in head/foot}%
    \usebeamerfont{date in head/foot}\insertshorttitle,~
    \insertshortdate{}\hspace*{1em}
    \insertframenumber{} / \inserttotalframenumber\hspace*{2ex}
  \end{beamercolorbox}}%
  \vskip0pt%
}
\beamertemplatenavigationsymbolsempty

\usepackage[percent]{overpic}
\usepackage{tikz}
%\usetikzlibrary{positioning,fit,shapes.arrows,shapes.geometric,shapes.misc,shapes.multipart,calc,shadows}
\tikzstyle{every picture}+=[remember picture]
\usepackage{booktabs}
\usepackage{graphicx}
\usepackage{rotating}
\usepackage{wasysym}
\usepackage{marvosym}
\graphicspath{{../../logo/}{figures/}{../../graphic-common/}}

\input{definitions.tex}
\newcommand{\lib}[1]{\tiny #1}

% Title etc
\title[G3G-Meeting]{Status Report from RA2/b}
\subtitle{\ZInvJets using \Zll and\\ \ttbar \wpj using \muonJets}
\author[Arne-Rasmus~Dr\"ager]{
  Arne-Rasmus~Dr\"ager (Uni Hamburg)\\on behalf of the RA2/b team
}
\date[October 31, 2014]{October 31, 2014
  \vskip1cm
  \begin{center}
    \includegraphics[height=1cm]{Universitaet-Hamburg-Logo.jpg}
    \hskip8cm
    \includegraphics[height=1cm]{CMSlogo.jpeg}
  \end{center}
}

% pdflatex packages
\hypersetup{bookmarks=true}
\hypersetup{unicode=false}
\hypersetup{pdftitle={Search for Higgs Bosons Beyond the Standard Model with the CMS Detector}}
\hypersetup{pdfauthor={Matthias Schr\"oder}}


\begin{document}
% ==================================================
% --------------------------------------------------
\begin{frame}
  \titlepage
\end{frame}

\section{Introduction}
% --------------------------------------------------
\begin{frame}
  \frametitle{Did We Discover the Standard-Model Higgs?}
  \begin{columns}
    \begin{column}{0.5\textwidth}
      \begin{itemize}
      \item General indications of physics beyond the Standard Model
        \begin{itemize}
        \item Dark matter
        \item Fine-tuning problem
        \end{itemize}
      \item Observed boson compatible with Standard-Model Higgs
      \item But from couplings analysis\footnote[frame]{presented on Monday by Shivali~Malhotra}:\\ \textbf{plenty of room for BSM Higgs decays!}
      \end{itemize}
    \end{column}
    \begin{column}{0.5\textwidth}
      \centering
      \begin{overpic}[width=0.9\textwidth]{figures/Closure__HT__MCPrMTWDiLep_vs_MCEx__csa_Baseline.pdf}
        %\put(19,45){\colorbox{white}{\red{$\text{BR}_{\text{BSM}} < 32\%$ at 95\% CL}}}
        \put(6,39){\colorbox{white}{\textcolor{red}{\bf BSM decays?}}}
        %\put(18,30){\textcolor{red}{$\longleftrightarrow$}}
  %      \put(17,30){\includegraphics[width=1.3cm]{aux/left-right-arrow_red.pdf}}
        \put(97,88){\rotatebox{-90}{\scriptsize CMS PAS HIG-14-009}}
        \put(13,2){\tiny assuming no tree-level modifications}
      \end{overpic}
    \end{column}
  \end{columns}
  \begin{block}{}
    \centering
    % SAY: 2 ways to proceed
    What are the properties of the new boson?\\
    Are there any additional Higgs bosons?
  \end{block}
\end{frame}

% --------------------------------------------------
\begin{frame}
  \frametitle{Examples of Models with Extended Higgs Sector}
  \begin{columns}[T]
    \begin{column}{0.7\textwidth}
      \textbf{Supersymmetry}
      \begin{itemize}
      \item Well-motivated extension of Standard Model
        \begin{itemize}
        \item[\greencheck] Provides dark-matter candidates
        \item[\greencheck] Solves fine-tuning problem
        \end{itemize}
      %\item Requires additional Higgs bosons
      \item Minimal supersymmetric extension (MSSM)
        \begin{itemize}
        \item 2 Higgs doublets $\rightarrow$ \textbf{5 physical bosons}\\
          \hskip3.1cm h, H, $\text{A}\equiv\Phi$ (neutral)\\
          \hskip3.1cm H$^{+}$, H$^{-}$ (charged)
        \item 2 tree-level parameters $m_{\text{A}}$ and $\tanbeta$
        \item h usually identified with 125\gev boson
        \end{itemize}
      \item Next-to-Minimal model (NMSSM)
        \begin{itemize}
        \item 2 doublets + 1 singlet = 7 physical bosons
        \end{itemize}
      \end{itemize}
      \textbf{Generic 2 Higgs-Doublet Models (2HDM)}
      \begin{itemize}
      \item Effective extension of Standard Model
      \item Allows flavour-changing Yukawa couplings
      \end{itemize}
    \end{column}
    \begin{column}{0.3\textwidth}
      \centering
      %\includegraphics[width=\textwidth]{figures/BUD_SPENCER_TERENCE_HILL_THUMBS_UP.jpg}\\
      \includegraphics[width=\textwidth]{figures/Closure__HT__MCPrMTWDiLep_vs_MCEx__csa_Baseline.pdf}
      \vskip0.2cm
      {\small MSSM predicts $m_{\text{h}}\lesssim135\gev$!}
    \end{column}
  \end{columns}
\end{frame}

% --------------------------------------------------
\begin{frame}
  \frametitle{Examples of BSM-Higgs Searches at CMS}
  \begin{center}
    \only<1>{\includegraphics[width=\textwidth]{figures/Closure__HT__MCPrMTWDiLep_vs_MCEx__csa_Baseline.pdf}}
    \only<2>{\includegraphics[width=\textwidth]{figures/Closure__HT__MCPrMTWDiLep_vs_MCEx__csa_Baseline.pdf}}
  \end{center}
\end{frame}

\section{Heavy Neutral Higgs \texorpdfstring{$\Phi\rightarrow\tau\tau$}{to 2 Tau-Leptons}}
%\subsection{\href{http://cds.cern.ch/record/1623367}{CMS PAS HIG-13-021}}
\subsection{\href{http://arxiv.org/abs/1408.3316}{submitted to JHEP (arXiv:1408.3316)}}
% --------------------------------------------------
\begin{frame}
  \frametitle{Search for a Heavy Neutral Higgs: $\Phi\rightarrow\tau\tau$}
  \begin{columns}
    \begin{column}{0.5\textwidth}
      \centering
      \begin{overpic}[width=\textwidth]{figures/Closure__HT__MCPrMTWDiLep_vs_MCEx__csa_Baseline.pdf}
        \put(30,81){\textcolor{green}{$\text{H}\rightarrow\text{bb}$}}
        \put(30,64){\textcolor{blue}{$\text{H}\rightarrow\tau\tau$}}
        \put(70,55){MSSM}
      \end{overpic}
      \begin{itemize}
      \item \textcolor{blue}{$\Phi\rightarrow\tau\tau$ channel}
        \begin{itemize}
        \item Relatively large BR
        \item Manageable backgrounds
        \end{itemize}
      \end{itemize}
    \end{column}
    \begin{column}{0.5\textwidth}
      \begin{itemize}
      \item \textbf{no b-tagged jet}
        \begin{overpic}[width=0.6\textwidth]{figures/Closure__HT__MCPrMTWDiLep_vs_MCEx__csa_Baseline.pdf}
        \end{overpic}
        \begin{itemize}
        \item gg-fusion production
        \item Dominant at small $\tanbeta$
        \end{itemize}
      \item \vskip0.3cm\textbf{$\geq1$ b-tagged jets}
        \begin{overpic}[width=0.6\textwidth]{figures/Closure__HT__MCPrMTWDiLep_vs_MCEx__csa_Baseline.pdf}
        \end{overpic}
        \begin{itemize}
        \item Associated production
        \item Dominant at larger $\tanbeta$
        \end{itemize}
      \end{itemize}
    \end{column}
  \end{columns}
\end{frame}

% --------------------------------------------------
\section{Lost Lepton Background }
\subsection{Basic concept}
\begin{frame}
\frametitle{Mainly \ttbar and \wpj events where prompt electrons and muons are lost}
 \begin{figure}
 \centering
  \includegraphics[width = 0.4\textwidth]{figures/lepton_veto_sketch.png}
%  \caption{Awesome figure}
 \end{figure}
\end{frame}

\begin{frame}
  \frametitle{Mainly \ttbar and \wpj events where prompt electrons and muons are lost}
   \begin{figure}
 \centering
  \includegraphics[width = 0.65\textwidth]{figures/lepton_veto_sketch.png}
%  \caption{Awesome figure}
 \end{figure}
      \begin{itemize}
      \item Select a control sample (CS) of exactly one well isolated $\mu$ within the acceptance
        \begin{itemize}
        \item Weight each CS event according to efficiencies for each identification step \\ (efficiencies determined from \ttbar and \wpj sample)
        \item Isolateion and reconstruction efficiency in \HT, \MHT and \NJets parametrized
        \item Acceptance in \MHT and \NJets
        \end{itemize}
      \end{itemize}
\end{frame}
% --------------------------------------------------
\subsection{\mt cut }
\begin{frame}
\begin{itemize}
 \item Signal and other SM processes can contribute to $\mu$ control sample
 \item Suppress contamination by requiring trans. mass $\mt < 100 \gev$ \\
\end{itemize}

\frametitle{}
  \begin{columns}
    \begin{column}{0.6\textwidth}
    \begin{centering}
     $m_{T} = \sqrt{2 \cdot p_{T}(\mu)\cdot \met (1 - \cos(\Delta \Phi))}$
\end{centering}
      \begin{itemize}
      \item Removes about 15\% of $\mu$ CS
        \begin{itemize}
        \item Most of them di-leptonic \ttbar decays
        \item Mismeasured jets
        \item Highly virtual W
        \end{itemize}
      \begin{centering}
      \begin{overpic}[width=0.7\textwidth]{figures/MuMTWMHTNjet.pdf}
      \end{overpic}
      \end{centering}
      \item Correction as a function of \MHT \NJets applied
      \end{itemize}
    \end{column}
    \begin{column}{0.4\textwidth}
      \centering
      \begin{overpic}[width=0.8\textwidth]{figures/ControlSample__MTW__MCPrMTWDiLepTTbar+MCPrMTWDiLepW__mu_control_sample.pdf}
        \put(41.52,83){\color{black}\line(0,-1){67}}
    %    \put(6,39){\colorbox{white}{\textcolor{red}{\bf BSM decays?}}}
    %    \put(97,88){\rotatebox{-90}{\scriptsize CMS PAS HIG-14-009}}
    %    \put(13,2){\tiny assuming no tree-level modifications}
      \end{overpic}
      \begin{overpic}[width=0.8\textwidth]{figures/2012_ControlSample__MTW__Data_vs_TTbar+WJets__Baseline.pdf}
        \put(41.52,83){\color{black}\line(0,-1){67}}
    %    \put(6,39){\colorbox{white}{\textcolor{red}{\bf BSM decays?}}}
    %    \put(97,88){\rotatebox{-90}{\scriptsize CMS PAS HIG-14-009}}
    %    \put(13,2){\tiny assuming no tree-level modifications}
      \end{overpic}
    \end{column}
  \end{columns}
\end{frame}
% --------------------------------------------------
\subsection{di leptonic contribution}
\begin{frame}
\begin{itemize}
 \item The $\mt$ cut reduces di leptonic \ttbar contribution to the single $\mu$ controlsample from ~5\% to about ~3\%
 \item But since the propability of losing two leptons is less likely than one di leptonic events are overestimated
 \item Separted estimation of lost di leptonic events applied
 \item Contribution to total amount of lost leptons ~1\%
\end{itemize}
  \begin{columns}
    \begin{column}{0.5\textwidth}
     \centering
      \begin{overpic}[width=0.95\textwidth]{figures/MuonDiLepMTW.pdf}
     \end{overpic}
    \end{column}
    \begin{column}{0.5\textwidth}
      \centering
      \begin{overpic}[width=0.95\textwidth]{figures/MuonDiLepEff.pdf}
      \end{overpic}
    \end{column}
  \end{columns}
\end{frame}
% --------------------------------------------------
\setcounter{framenumber}{24}

\end{document}
